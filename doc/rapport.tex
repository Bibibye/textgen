\documentclass[a4paper,10pt]{article}
\usepackage{url}
\usepackage[francais]{babel}
\usepackage[utf8]{inputenc}
\usepackage[T1]{fontenc} 
\usepackage{lmodern}
\usepackage{fontenc}
\usepackage[normalem]{ulem}
\usepackage{enumerate}
\usepackage{color}
\usepackage{subfig}
\usepackage{csquotes}
\usepackage{hyperref}
\usepackage{verbatim}
\usepackage{graphicx}
\usepackage{listings}
\usepackage{soul}
\usepackage{todonotes}
\usepackage{amsmath}
\usepackage{amssymb}


\newcommand\Nom\textsc
\newcommand\Terme\textit
\newcommand\Code\texttt
\newcommand\Type\hl

\title{
	\textbf{Web sémantique, langue et raisonnement :\\
          Générateur de phrases} }

\author{
	Joris \Nom{Pablo} \and Raphaël \Nom{Gaudy} }


\begin{document}
\maketitle

\section*{Introduction}
Ce projet, réalisé dans le cadre de l'unité d'enseignement \Terme{Web sémantique, langue et raisonnement}, est un générateur de phrases en français écrit avec \Terme{Prolog}. Ce rapport détaille l'ensemble des phénomènes grammaticaux présents dans les phrases générées, ainsi que leur implémentation.

\section{Phénomènes grammaticaux choisis}
\begin{itemize}
\item \textbf{Accord en genre, nombre et personne.} Tous les adjectifs qualificatifs sont accordés correctement, et les verbes conjugués à la bonne personne. 
  \begin{align*}
    \text{Exemple :~} \textit{Les grandes maroufles marchent}
  \end{align*}

\item \textbf{Valence verbale.} Les verbes sont complétés selon leur valence. 
  \begin{align*}
    \text{Exemples :~} 
    &\textit{Je marche}\\
    &\textit{Je mange un gnou poilu}\\
    &\textit{Je semble manger un gnou poilu}\\
    &\textit{Je dis que je mange un gnou poilu}
  \end{align*}

\item \textbf{Position de l'adjectif.} Les adjectifs ne peuvent pas être placés n'importe où.
  \begin{align*}
    \text{Exemples :~}
    &\textit{Un grand gnou}\\
    &\textit{Un gnou poilu}
  \end{align*}

\item \textbf{L'élision.} Certains mots du lexique ont une forme élidée de manière à éviter les hiatus.
\begin{align*}
  \text{Exemples :~}
  &\textit{Je mange}\\
  &\textit{J'entends manger le gnou}
\end{align*}

\item \textbf{La coordination.} Deux phrases peuvent être accolées grâce à des conjonctions de coordination ou locutions conjonctives.
\begin{align*}
\text{Exemple :~}
\textit{Je marche tandis que je mange un gnou}
\end{align*}
\end{itemize}

Pour tous les phénomènes grammaticaux pouvant être enchassés, nous avons choisi de limiter la récursion à $1$ (au plus un adjectif avant et un après un nom commun, au plus deux phrases accolées, au plus une seule subordonnée complétive). De cette manière, les phrases restent raisonnablement grandes et sont plus naturelles, et le générateur en produit un nombre fini.

\section{Implémentation}
\subsection{Lexique}
Le fichier \texttt{lexique.pl} contient l'ensenble des mots utilisés par notre générateur pour produire ses phrases, ainsi qu'un ensemble de prédicats permettant d'extraire des informations sur ces mots. Un mot est un fait de la forme :
\begin{verbatim}
lex(mot,nature,nombre,genre,personne,valence,position,debut,forme).
\end{verbatim}
Avec :
\begin{itemize}
\item \texttt{mot} le mot dont il est question.
\item \texttt{nature} la nature du mot (\texttt{nc} pour nom commun, \texttt{det} pour déterminant etc \dots).
\item \texttt{nombre} et \texttt{genre} pouvant être respectivement \texttt{plur} ou \texttt{sing} et \texttt{masc} ou \texttt{fem}.
\item \texttt{personne} un entier appartenant à $ \{1,2,3\} $.
\item \texttt{valence} la valence des verbes (\texttt{n0}, \texttt{n1}, \texttt{comp}, \texttt{vcomp} etc \dots).
\item \texttt{position} les positions possibles pour un adjectif (\texttt{post}, \texttt{pre} ou \texttt{both}).
\item \texttt{debut} correspondant au premier son du mot (\texttt{voy} pour voyelle et \texttt{con} pour consonne).
\item \texttt{forme} indiquant s'il s'agit d'une forme élidée ou non (\texttt{voy} pour la forme élidée, de manière à faire la correspondance avec les mots commençant par un son voyelle).
\end{itemize}

Pour obtenir plus facilement les informations relatives à un mot (sans utiliser \texttt{lex/9}), un certain nombre de prédicats ont été définis :
\begin{itemize}
\item \texttt{est(Mot,Nature)}
\item \texttt{accord(Mot,Nature,Genre,Personne)}
\item \texttt{valence(Mot,Valence)}
\item \texttt{position(Mot,Position)}
\item \texttt{debut(Mot,Debut)}
\item \texttt{forme(Mot,Forme)}
\end{itemize}

\subsection{Grammaire}
Le fichier \texttt{grammaire.pl} contient notre grammaire. Chaque symbole est paramétré par son sous-arbre syntaxique ainsi qu'un ensemble d'informations le concernant. Les terminaux de la grammaire sont définis de manière à pouvoir changer de lexique facilement. Par exemple, le symbole \texttt{v/7} :
\begin{verbatim}
v(v(Mot),N,P,V,D) --> [Mot], {est(Mot, v),
			      accord(Mot, N, _, P),
			      valence(Mot, V),
			      debut(Mot,D)}.
\end{verbatim}

Pour gérer la récursivité de certaines règles, nous avons ajouté un paramètre supplémentaire, comme pour le symbole de départ par exemple.
\begin{verbatim}
s(s(T1,T2),Enchassement,_,D) --> 
    np(T1,N,P,_,D,F), 
    vp(T2,N,P,F,Enchassement).

s(s(T1,T2,T3),Enchassement,1,D) --> 
    s(T1,Enchassement,0,D), 
    cc(T2,F), 
    s(T3,Enchassement,0,F).
\end{verbatim}

Le $2^e$ argument ici correspond à l'enchassement de subordonnées complétives, et le $3^e$ à la juxtaposition de phrases (s'il vaut $1$, on peut juxtaposer $2$ phrases pour lesquelles il vaudra $0$). La valeur de \texttt{Enchassement} est transmise jusqu'au syntagme verbal où se fait le choix de compléter ou non le verbe.

\subsection{Tester le travail}
Pour tester notre travail, nous avons ajouté un fichier \texttt{main.pl} pouvant être exécuté directement, sans arguments :
\begin{verbatim}
./main.pl
\end{verbatim}
Il affiche alors toutes les phrases engendrées par notre grammaire avec notre lexique. Cependant, cette manière de procéder a ses limites, et si le lexique est trop gros, ou s'il y a des règles engendrant trop de phrases dans la grammaire, alors la pile \Terme{Prolog} explose. Pour cette raison, une grande partie du lexique est commentée, ainsi que la seconde règle pour le symbole \texttt{s/6} de notre grammaire (Pour une grammaire engendrant $n$ phrases, ajouter cette règle et $m$ conjonctions de coordination fait engendrer $n^2\times m$ phrases).

Ce problème est dû au fait que l'ensemble des phrases est généré avant d'être affiché. Une approche plus impérative permettrait de résoudre ce problème.

Pour tester l'ensemble de la grammaire, il faut donc décommenter les lignes commençant par un '\%' simple dans les fichiers \texttt{lexique.pl} et \texttt{grammaire.pl}. Il faut alors ensuite utiliser un interprète \Terme{Prolog} et jouer avec la structure de l'arbre syntaxique pour obtenir les phrases qui nous intéressent.

\end{document}

%%% Local Variables: 
%%% mode: latex
%%% TeX-master: t
%%% End: 
